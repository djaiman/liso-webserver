\label{sec:liso}

In this section, we give an overview of the the complete requirements for the Liso server: what you will turn in at the final checkpoint. 
In the next three sections (\S\ref{sec:cp1}--\ref{sec:cp3}), we break the project into a development strategy with intermediate 
checkpoints.  The starting point for your server is framework code that we will provide on the course website and in Autolab. 


\subsection{Supporting HTTP 1.1}
You will find in RFC 2616 that there are many HTTP methods: commands that a client sends to a server.
Liso will only support three methods:

\begin{itemize}
\item \textbf{GET} -- requests a specified resource; it should not have any
                       significance other than retrieval
\item \textbf{HEAD} -- asks for an identical response as GET, without the actual
                       body---no bytes from the requested resource
\item \textbf{POST} -- submit data to be processed to an identified resource;
                       the data is in the body of this request; side-effects
                       expected
\end{itemize}

\noindent{\bf Error Messages:}
For all other commands, your server must return ``501 Method Unimplemented.'' 
If
you are unable to implement one of the above commands (perhaps you ran out of
time), your server must return the error response ``501 Method Unimplemented,''
rather than failing silently (or not so silently).  

\vspace{5pt}

\noindent{\bf Robustness:} As a public server, your implementation should be
robust to client errors. 
Even if the client sends malformed inputs or breaks off the connection mid-request, your server should never crash.
For example, your server must not
overflow any buffers when a malicious client sends a message that is ``too long." 
The starter code we will provide you is not robust to malformed requests (e.g, it cannot handle requests which do not have proper [CR][LF] line endings) and so you will have to extend it.

\vspace{5pt}

\noindent{\bf Multiple Requests:} During a given connection, a client may send multiple HEAD/GET/POST requests.
The client may even send multiple requests back-to-back, without even waiting for a response.
This is called `HTTP pipelining'~\cite{pipelining}. Your server must support multiple requests in the same connection, and it must support HTTP pipelining.

\subsection{Supporting Many Clients}
Your server should be able to support multiple clients concurrently. The only
limit to the number of concurrent clients should be the number of available
file descriptors in the operating system (the min of ulimit -n and
FD\_SETSIZE---both typically 1024 but it may be higher or lower).  


While the server is waiting for a client to
send the next command, it should be able to handle inputs from other clients.
Clients may `stall' (send half of a request and then stall before sending the rest) or cause errors; these problems should not harm other concurrent users.
For example, if a client only sends half of a request and stalls, your server should move on to serving another client.
In general, concurrency can be achieved using either \texttt{select()} or
multiple threads. However, in this project, you \textbf{must implement your
server using \texttt{select()} to support concurrent connections}. Threads are
\textbf{NOT} permitted at all for the project. 

\subsection{HTTPS and CGI}
Students enrolled in 15-641 will extend your basic Liso server with support for HTTPS (which encrypts connections to your server) and CGI (which allows your server to support interactive programs) in Checkpoint 3.

\vspace{5pt}

\noindent{\bf HTTPS Support:}
Your server will use the OpenSSL library for HTTPS support.  Specially, you will wrap communication calls with with SSL wrapping functions that will encrypt data sent over the channel and authenticate the server based on a certificate.  
In order to test HTTPS, you will need to get a certificate that your HTTPS-enabled server can use to authenticate itself to clients.  

\noindent{\bf CGI Support:} The Common Gateway Interface (CGI) provides a standard interface that allows a web server to call other processes or servers.  Web servers use this typically to generate dynamic content, i.e., the content that is generated is based on input provided by the client (using POST) or other client-specific information. 
Your server will support CGI requests and will provide a built in `ASCII art generator' as a demo application.
When a client submits a POST to the ASCII art generator, the server will generate a page that spells out client-specified phrase in ASCII art.

\vspace{5pt}

\noindent More details on HTTPS and CGI are in \S\ref{sec:cp3}.
\subsection{Command Line Arguments}
\noindent\textbf{Liso will always have 8 arguments. You may not modify Liso to, e.g., to take a different number of arguments or to reorder the arguments. If you do not need one of the arguments in the earlier checkpoints, you may simply ignore it in your code.}\\

\textbf{\emph{usage:}} \emph{./lisod $<$HTTP port$>$ $<$HTTPS port$>$
                             $<$log file$>$ $<$lock file$>$ $<$www folder$>$
                             $<$CGI script path$>$ $<$private key file$>$ $<$certificate file$>$}

\begin{description}
	\item[\textnormal{\emph{HTTP port}}] -- the port for the HTTP (or echo) server
                                          to listen on

	\item[\textnormal{\emph{HTTPS port}}] -- the port for the HTTPS server to
                                           listen on

	\item[\textnormal{\emph{log file}}] -- file to send log messages to
                                         (debug, info, error)

	\item[\textnormal{\emph{lock file}}] -- file to lock on when becoming a daemon
                                          process

	\item[\textnormal{\emph{www folder}}] -- folder containing a tree to serve as
                                           the root of a website

	\item[\textnormal{\emph{CGI script name (or folder)}}] --
										   for this project, this is a file that should be
										   a script where you redirect all
										   /cgi/* URIs. In the real world, this
										   would likely be a directory of
										   executable programs.

	\item[\textnormal{\emph{private key file}}] -- private key file path

	\item[\textnormal{\emph{certificate file}}] -- certificate file path
\end{description}


\subsection{Tools, Skills, and Grading}
The projects in this course are significant larger and more complex
than the 15-213/513 projects.  They are also more open-ended, in the
sense that we only specify \emph{what} the system must do, but not
\emph{how} you do it.  For example, a web server needs to perform several 
functions, e.g., managing sessions, using the file system, generating
responses including error messages, etc.  You are responsible for the 
design: what modules you have, the data structures they use, and how they
interaction.

You will also have to strengthen your programming skills.  A big part
of that is making good use of of a variety of tools, such as {\tt
gdb}, {\tt Valgrind}, and others (see Section~\ref{sec:impl}.  We will
review these tools in recitations sessions early on in the semester.
One important skill is debugging.  You should first try to debug code
yourself (using the above tools) but if you have problems, you 
use the office hours of not only the lead TA but also the
other TAs to get help.  Note that the TAs are not allowed to debug
your code for you. Specifically, they will only look at your code for
a limited time (up to 10 minutes; leaving them time to help other
students) and they will not modify your code (they cannot touch the
keyboard).

You will also have to develop your own test suite to test and help in
debugging your code.  While we will give you some tests for each
checkpoint, these tests will only test some of the features of your
implementation and they are only a subset of the tests we will use for
grading.  You are responsible for writing additional tests so you
cover all the features and requirements listed in this handout.


