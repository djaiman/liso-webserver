\label{sec:cp2}
In this checkpoint, your web server will reply to client requests as a real HTTP web server for static content. You will be able to browse to your server using a real web browser!

\vspace{5pt}

\noindent Your server MUST:
\begin{itemize}
  \item Respond to properly formatted HTTP HEAD and GET requests. You do not need to respond to POST requests yet (continue to echo reply to POST requests if the request is correctly formatted; send an error 400 if the request is malformed).
    \item Support three HTTP 1.1 error codes: 400, 501, and 505. 501 is for unsupported methods; 505 is for bad version numbers. Everything else can be handled with a 400.
    \item Handle concurrent connections using select().
    \item Handle pipelined requests.
\end{itemize}

\subsection{Getting Started}
\begin{enumerate}
\item Begin with your repository for Checkpoint 2, using the result of Checkpoint 1 as a starting point.
\item You are highly encouraged to create a simplified logging module for your project that writes out formatted logs to the log file specified on the command-line; this will help you debug and trace requests that come to your system. {\it Nonetheless we will not require this in grading.} See \href{https://httpd.apache.org/docs/1.3/logs.html\#common}{here} for some examples of how Apache handles logging.
\item Enhance your server to respond properly to any HTTP 1.1 request and implement persistent connections with HEAD and GET as defined in RFC 2616. At this point as we don't have CGI and so it doesn't make sense to support POST requests -- continue to echo reply to correctly formatted POSTs.
\item Your server will need to handle lots of concurrent and pipelined requests: make sure you test with many simultaneous connections. Your server should respond to pipelined requests in the order that they were received.
\item The server should also handle errors in a practical way. {\bf It should never completely crash} (make it as robust as possible). In testing, try to crash your server by sending malformed and strange requests and fix your server to prevent these crashes.
\item Double check that your server sends the appropriate error messages to malformed requests.
\item Submission is the same as Checkpoint 1. Tag and upload your repo in a tarball to the corresponding lab on Autolab.
\end{enumerate}


\subsection{Tips and FAQ}
\begin{enumerate}
  \item Do we have to include a Last-Modified field in our responses?\\
    \textbf{Yes}
    \item Do we have to support Chunking?\\
    \textbf{No}
    %\item For POST, if a Content-Length header is not provided in the request, is it fine to respond with a 411 as specified in the RFC?\\
    %\textbf{Yes}, this is both possible and desired because it is a valid response. It may also simplify the design of the server.
    \item Do we have to support "Conditional" GETs?\\
    \textbf{No.}
    \item Can I use a hash table library written by another person?\\
    \textbf{No}, for this project implement your own if you really want it. You won't have to track every header, only the important ones for basic compliance.
    \item Should we expect HTTP/1.1 requests to fit in one buffer?\\
    \textbf{No}, do not assume that they always fit inside one buffer. Be prepared to parse across buffer boundaries.
    \item Can I assume that the request always has the Content-length field?\\
    \textbf{Yes}, assume requests to your server have Content-length if applicable. If it is missing, return a 400 response.
    \item Since the server will serve static files, are we allowed to use http://svn.apache.org/repos/asf/httpd/httpd/trunk/docs/conf/mime.types\\
    \textbf{Yes}, You are allowed to use that or a simplified version of it.No need to support all well-known MIME-types, just the most common ones: text/html, text/css, image/png, image/jpeg, image/gif and maybe a few others up to your discretion.
    \item For last-modified field, is there a C library that we can read file
     metadata  such as this? Or is our web server supposed to handle that
     manually. (i.e. stamping each and every file in the www root with a
     last-modified stamp and manually updating that stamp every time a file is
     changed....).\\
    \textbf{stat()} is a system call to check for metadata on a file. 
    \item Are we allowed to reject requests with headers beyond a maximum size?\\
    \textbf{Yes}, you may reject any header line larger than 8192 bytes. Note, that this is different than a Content Length of greater than 8192. Additionally, you must find and parse the next request properly for pipelining purposes if you do reject the request.
    \item Should we also handle request made up with /n instead of /r/n?\\
    Some web servers do this and it is nice for telnet testing. However, Liso does not have this as a requirement -- you do not have to do this.
\end{enumerate}

\subsection{Grading}

The breakdown of grading for Checkpoint 2 is below.
%
\begin{center}
  \begin{tabular}{>{\centering\arraybackslash}m{1in}>{\centering\arraybackslash}m{1in}p{3in}}
  {\bf Task}&{\bf Weight}&{\bf Subcriteria}\\
  \hline
  \addlinespace[5pt]
    Format&10\%&\vspace{-10pt} 
                              {\it Assigned by human grader:}
                              \begin{packed_itemize}
                                \item Correct turnin -- Makefile, compilation, properly tagged repo, TAs don't have to edit or search for files (10\%)
                                \item Code style (5\%)
                                \item Code commenting (5\%)
                              \end{packed_itemize}\\

  \hline
  \addlinespace[5pt]

    Error Handling&15\%&\vspace{-10pt} 
                              {\it Assigned in AutoLab:}
                              \begin{packed_itemize}
                                \item Responds to wrong version with 505
                                \item Responds to unsupported method with 501
                                \item Responds to other problems / invalid requests with 400.
                              \end{packed_itemize}\\

  \hline
  \addlinespace[5pt]

  Request Handling&45\%&\vspace{-10pt} 
                              {\it Assigned in AutoLab:}
                              \begin{packed_itemize}
                                \item Replies to correctly formatted HEAD requests.
                                \item Replies to correctly formatted GET requests.
                                \item Echos in reply to correctly formatted POST requests.
                                \item Accepts and responds to pipelined requests.
                              \end{packed_itemize}\\

  \hline
  \addlinespace[5pt]

  System Design&30\%&\vspace{-10pt} 
                              {\it Assigned in AutoLab:}
                              \begin{packed_itemize}
                                \item Never crashes.
                                \item Uses select() (no multithreading).
                                \item Handles hundreds of concurrent connections when tested with Apache Bench.
                              \end{packed_itemize}\\

\end{tabular}
\end{center}


\noindent Checkpoint 2 is worth 25\% of P1, which is worth 15\% of your final grade. Hence, in completing Checkpoint 2 with a perfect score you would earn 3.75\% of the points available in the final grade.

