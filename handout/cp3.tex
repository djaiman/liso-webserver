\label{sec:cp3}
In this checkpoint, students enrolled in 15-641 will add support for
CGI requests and for HTTPS, which uses encryption to keep connection
content private from Internet eavesdroppers.  In addition, you will
`daemonize' your server to allow it to run as a persistent background
task, and you will double-check all of the requirements from
Checkpoints 1 and 2 for a final grade.

\vspace{5pt}
\noindent Your server MUST:
\begin{itemize}
  \item Support the CGI standard as specified in RFC 3875.
  \item Support HTTPS via TLS as specified in RFC 2818.
  \item Run as a daemonized, background process.
  \item {\bf Remember: Continue to support regular HTTP requests on port 80, as in CP2.}
\end{itemize}

\subsection{Getting Started}
We suggest you work on this checkpoint in two phases. In one phase, implement CGI. In another phase, add support for HTTPS. You can do these steps in either order -- just don't try to do both at the same time!  The remainder of this section explains the steps involved in completing CP3.  A high-level overview of
CGI and HTTPS will be presented in the web lecture in the beginning of the semester (before you get to CP3).


\subsubsection{Implementing CGI}
The Common Gateway Interface (CGI) allows a web browser to generate dynamic content for a client, e.g., content that is based on input the client provided.  
When the web server receives a request using a CGI URI, it will start a new process in which it starts the CGI program and provides it with the client's requests.  
CGI program then generates the response and returns it to the web server, which returns it to the client.
We provide instructions for implementing CPI below, but we recommend that you `skim' the CGI RFC~\cite{cgirfc} and the URI RFC~\cite{urirfc} to get 
the big picture before diving in to your implementation.


\begin{enumerate}
  
  \item Create support for the CGI variables listed in \S\ref{sec:cgi-vars}. Not all variables may be used in the ASCII Art sample, but we will have additional checks to make sure that you do set these variables in case other CGI scripts are used. Using a python CGI script that echos the environment variables may be useful for your debugging.
\item Implement your CGI module. Any URI starting with ``/cgi/'' will be handled by a single command-line specified executable via a CGI interface coded by you. We have also provided a CGI runner in C. 
\begin{enumerate}
    \item CGI URI's may accept GETs, POSTs, and HEADs; your job is not to decide this, just pass along information to the program being called
    \item You need to pipe stdin, pipe stdout, fork(), setup environment variables per the CGI specification, and execve() the executable (it should be executable) Note: Watch the piped fd's in the parent process using your select() loop. Just add them to the appropriate select() sets and treat them like sockets, except you have to pipe them further to specific sockets.
    \item Pass any message body (especially for POSTs) via stdin to the CGI executable
    \item Receive any response over stdout until the process dies (monitor process status), or there is nothing more to read or a broken pipe is encountered
    \item If the CGI program fails in any way, return a 500 response to the client, otherwise send all bytes from the stdout of the spawned process to the requesting client.
    \item The CGI application will produce headers and message body as it sees fit, you do not need to modify or inspect these bytes at all.
\end{enumerate}
\end{enumerate}

\subsubsection{Implementing HTTPS}
HTTPS uses the Transport Layer Security (TLS) protocol to secure HTTP sessions.  TLS is a session layer protocol so it sits between the HTTP protocol you are implementing
and the TCP transport protocol.  TLS ensure the \emph{secrecy} and \emph{integrity} of the data exchanged over the TCP connection, so third parties cannot read or modify 
the data.  TLS also allows the client to \emph{authenticate} web server, the client knows it is communicating with the right server.  Authentication is based on a 
\emph{certificate}, a data structure links the server's URL to its public key and is signed by a Certificate Authority (CA).  To support HTTPS, your web server 
needs to use the OpenSSL library instead of the socket API, and the person responsible for the web server (that means you) needs to obtain a URL and a signed certificate.

Please follow the following instructions to implement HTTPS:
\begin{enumerate}
\item Create a DNS hostname for yourself with a free account at No-IP \cite{WWW:no-ip} (or use a domain name you already have...)
\item In order to test your HTTPS implementation, we need to monitor your (encrypted!) HTTPS traffic.  To allow this, add the 15-441 Carnegie Mellon University Root CA to your browser (import certificate, usually somewhere in preferences)
\begin{enumerate}
    \item Now we can man-in-the-middle your HTTPS :-) Being course staff has perks!
    \item But just trust us till this part is over...or make your own \textbf{\href{http://www.g-loaded.eu/2005/11/10/be-your-own-ca/}{CA}}.
    \item Really though, this is the part of the course where you need to Reflect on Trusting Trust.
\end{enumerate}
\item Obtain your own private key and public certificate from the 15-441 CMU CA.  You will need this when you run your HTTPS-enabled web server.
\item Implement SSL support - we have provided you a sample C server in the Autolab Handout. 
\begin{enumerate}
    \item Use the OpenSSL library. \cite{www:openssl}
    \item Create a second server listening socket in addition to the first one. Use the passed in SSL port from the commandline arguments.
    \item Add this socket to the select() loop just like your normal HTTP server socket.
    \item Whenever you accept connections, wrap them with the SSL wrapping functions.
    \item Use the special read() and write() SSL functions to read and write to these special connected clients
    \item If you setup your browser, you may now verify that connections to your webserver use TLSv1.0; inspect the ciphers, message authentication hash scheme, and key exchange methods used by your server.
\end{enumerate}
\item Implement daemonization - we have provided you a sample of daemonizing code.
\item Submission is the same as Checkpoint 1. Tag and upload your repo in a tarball to the corresponding lab on Autolab.
\end{enumerate}

\subsection{Grading}
%
The breakdown of grading for Checkpoint 3 is below.
%
\begin{center}
  \begin{tabular}{>{\centering\arraybackslash}m{1in}>{\centering\arraybackslash}m{1in}p{3in}}
  {\bf Task}&{\bf Weight}&{\bf Subcriteria}\\
  \hline
  \addlinespace[5pt]
    Format&10\%&\vspace{-10pt} 
                              {\it Assigned by human grader:}
                              \begin{packed_itemize}
                                \item Correct turnin -- Makefile, compilation, properly tagged repo, TA's don't have to edit or search for files (10\%)
                                \item Code style (5\%)
                                \item Code commenting (5\%)
                              \end{packed_itemize}\\

  \hline
  \addlinespace[5pt]

    CGI&30\%&\vspace{-10pt} 
                              {\it Assigned in AutoLab:}
                              \begin{packed_itemize}
                                \item Supports all CGI variables
                                \item Runs provided ASCII Art CGI script
                                \item Accepts/Replies to CGI requests from multiple clients at the same time over HTTP.
                              \end{packed_itemize}\\

  \hline
  \addlinespace[5pt]

  HTTPS&20\%&\vspace{-10pt} 
                              {\it Assigned in AutoLab:}
                              \begin{packed_itemize}
                                \item Presents a certificate and initiates a TLS connection.
                                \item GET, HEAD, and POST requests all work through the HTTPS connection.
                                \item Multiple clients can connect at the same time over HTTPS.
                              \end{packed_itemize}\\

  \hline
  \addlinespace[5pt]

  Basic HTTP&40\%&\vspace{-10pt} 
                              {\it Assigned in AutoLab:}
                              \begin{packed_itemize}
                                \item All criteria from CP2.
                              \end{packed_itemize}\\

\end{tabular}
\end{center}


\noindent Checkpoint 3 is worth 60\% of P1, which is worth 15\% of your final grade. Hence, in completing Checkpoint 3 with a perfect score you would earn 9\% of the points available in this class.

