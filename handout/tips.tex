
\section{Getting Started}
This section gives suggestions for how to approach the project. Naturally, other
approaches are possible, and you are free to use them.

\begin{itemize}
	\item \textbf{Start early!} The hardest part of getting started tends to be
	  getting started. Remember the 90-90 rule: the first 90\% of the job takes
	  90\% of the time; the remaining 10\% takes the other 90\% of the time.
	  Starting early gives you time to ask questions. For clarifications on
	  this assignment, post to Piazza and read project updates on the course
	  web page.  Talk to your classmates. While you need to write your own
	  original program, we expect conversation with other people facing the
	  same challenges to be very useful. Come to office hours. The course staff
	  is here to help you.

	\item Read the RFCs selectively. RFCs are written in a style that you may
      find unfamiliar. However, it is wise for you to become familiar with it,
      as it is similar to the styles of many standards organizations. We don't
      expect you to read every page of the RFC, especially since you are only
      implementing a small subset of the full protocol, but you may well need to
      re-read critical sections a few times for the meaning to sink in.

	\item Begin by taking a cursory first pass over the RFCs. Do not focus on
      the details; just try to get a sense of how they work at a high
      level. Understand the role of the server. Understand what error conditions
      are possible, and how they are used. You may want to print the RFCs, and
      mark them up to indicate which parts are important for this project, and
      which parts are not needed. You may need to re-read these sections several
      times.

	\item Next, take a second pass over the RFCs. You will want to read all of
      them together.  Again, do not focus on the details; just try to understand
      the requests and responses at a high level. As before, you may want to
      mark up a printed copy to indicate which parts of the RFCs are important
      for the project, and which parts are not needed.

	\item Now, go back and read with an eye toward implementation. Mark the
      parts which contain details that you will need to write your server.
      Start thinking about the data structures (input and output buffers, etc.)
      your server will need to maintain. What information needs to be stored
      about each client while servicing requests (maybe an HTTP 1.1 finite state
      machine per client, etc.)?

	\item Get started with a simple server that accepts connections from
	  multiple clients. It should take any message sent by any client, and
	  ``echo'' that message back to its sender.  This server will not be
	  compatible with HTTP clients, but the code you write for it will be
	  useful for your final server. Writing this simpler server will let you
	  focus on the socket programming aspects of a server without worrying
	  about the details of the protocols. Test this simple server with the provided Python script in Checkpoint 1.

    \item Next, enhance the starter code we have provided for HTTP parsing. Apply all the RFC knowledge you have gathered from previous steps and try to convert them into rules. After you combine this parser along with the echo server, you should be ready for Checkpoint 1. 
    
	\item At this point, you are ready to write a standalone HTTP 1.1
      server. But do not try to write the whole server at once. Decompose the
      problem so that each piece is manageable and testable.  For each request,
      identify the different cases that your server needs to handle. Find common
      tasks among different commands and group them into procedures to avoid
      writing the same code twice. You might start by implementing the routines
      that read and parse commands. Then implement commands one by one, testing
      each with \texttt{telnet}.

	\item Thoroughly test your server. Use the provided scripts to test basic
	  functionality. For further testing, use \texttt{telnet}, a web browser,
	  or replay scripts. Learn Python from our scripts and as we go to make
	  repeatable ``regression tests''---every time you implement a new feature
	  you use regression tests to see if anything broke.

	\item Make sure to check the return code of all system calls and handle
      errors appropriately. Temporary failures (e.g., EINTR) should not cause
      your server to abort or exit in failure.  Fatal errors can be dealt with
      via a perror() call and exiting---but try to clean up open file descriptors
      and sockets nicely even when fatally exiting.

	\item Be liberal in what you accept and conservative in what you
      send~\cite{RFC:1122}.  Following this guiding principle of Internet design
      will help ensure your server works with many different and unexpected
      client behaviors.

	\item Code quality is important. Make your code modular and extensible where
      possible. You should probably invest an equal amount of time in testing
      and debugging as you do writing. Also, debug incrementally.  Write in
      small pieces and make sure they work before going on to the next piece.
      Your code should be readable and commented. Not only should your code be
      modular, extensible, readable, etc, most importantly, it should be your
      own!

    \item You may want to consider turning warnings into errors to avoid bad
	  programming style. Do this by passing \texttt{-Werror} to \texttt{gcc}
	  during compilation.

	\item If you have a question about a project handout or a technical issue,
	  there is an excellent chance that other students have the same question.
	  Please read Piazza to see if there has been traffic and consider posting
	  your questions there.

\end{itemize}

\section{Resources}
For information on network programming, the following may be helpful:

\begin{itemize}
    \item Beej's Guide~\cite{beej}
	\item Class Textbook -- Sockets, etc
	\item Class Piazza --  Announcements, clarifications, etc
	\item Class Website --  Announcements, errata, etc
	\item Computer Systems: A Programmer's Perspective (CS 15-213 text book)\cite{Book:CSAPP}
	\item BSD Sockets: A Quick And Dirty Primer\cite{BSD:Sockets}
	\item An Introductory 4.4 BSD Interprocess Communication Tutorial\cite{BSD:IPC}
	\item Unix Socket FAQ\cite{FAQ:Sockets}
	\item Sockets section of the GNU C Library manual
		\begin{itemize}
			\item Installed locally: info libc
			\item Available online: GNU C Library manual\cite{Manual:libc}
		\end{itemize}
	\item man pages
		\begin{itemize}
			\item Installed locally (e.g. man socket)
			\item Available online: the Single Unix Specification\cite{Spec:Unix}
		\end{itemize}
	\item Other Google Groups / Stackoverflow - Answers to almost anything\cite{Google:Groups}
\end{itemize}


