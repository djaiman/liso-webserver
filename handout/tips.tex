This section gives suggestions for how to approach the project. Naturally, other
approaches are possible, and you are free to use them.

\subsection{Start Early}
	The hardest part of getting started tends to be
	  getting started. Remember the 90-90 rule: the first 90\% of the job takes
	  90\% of the time; the remaining 10\% takes the other 90\% of the time.
	  Starting early gives you time to ask questions. For clarifications on
	  this assignment, post to Piazza and read project updates on the course
	  web page.  Talk to your classmates. While you need to write your own
	  original program, we expect conversation with other people facing the
	  same challenges to be very useful. Come to office hours. The course staff
	  is here to help you.

\subsection{How to read an RFC}
	 
   Read the RFCs selectively. RFCs are written in a style that you may
      find unfamiliar. However, it is wise for you to become familiar with it,
      as it is similar to the styles of many standards organizations. We don't
      expect you to read every page of the RFC, especially since you are only
      implementing a small subset of the full protocol, but you may well need to
      re-read critical sections a few times for the meaning to sink in.

	 Begin by taking a cursory first pass over the RFCs. Do not focus on
      the details; just try to get a sense of how they work at a high
      level. Understand the role of the server. Understand what error conditions
      are possible, and how they are used. You may want to print the RFCs, and
      mark them up to indicate which parts are important for this project, and
      which parts are not needed. 

	 Next, take a second pass over the RFCs, focusing on the sections that 
      that describe functionality you need to implement. You will want to read all of
      them together.  Again, do not focus on the details; just try to understand
      the requests and responses at a high level. As before, you may want to
      mark up a printed copy to indicate which parts of the RFCs are important
      for the project, and which parts are not needed.  

	 Before you start coding, you then go back and read with an eye toward implementation. Mark the
      parts which contain details that you will need to write your server code.
      You may want to add bookmarks to the sections that you will need to reference during the
      implementation so you can find them quickly.
      Start thinking about the data structures (input and output buffers, etc.)
      your server will need to maintain. What information needs to be stored
      about each client while servicing requests (maybe an HTTP 1.1 finite state
      machine per client, etc.)?

\subsection{Testing}
	Thoroughly test your server. Use the provided scripts to test basic
	  functionality. For further testing, use \texttt{telnet}, a web browser,
	  or replay scripts. Learn Python from our scripts and as we go to make
	  repeatable ``regression tests''---every time you implement a new feature
	  you use regression tests to see if anything broke.

    \vspace{5pt}
   \noindent{\bf Make sure to check the return code of all system calls and handle
      errors appropriately.} Temporary failures (e.g., EINTR) should not cause
      your server to abort or exit in failure.  Fatal errors can be dealt with
      via a perror() call and exiting---but try to clean up open file descriptors
      and sockets nicely even when fatally exiting.

    \vspace{5pt}
   \noindent{\bf Be liberal in what you accept and conservative in what you
      send~\cite{RFC:1122}.}  Following this guiding principle of Internet design
      will help ensure your server works with many different and unexpected
      client behaviors.

    \vspace{5pt}
    
    \noindent{\bf Turn warnings into errors.} You may want to consider turning warnings into errors to avoid bad
	  programming style. Do this by passing \texttt{-Werror} to \texttt{gcc}
	  during compilation.


