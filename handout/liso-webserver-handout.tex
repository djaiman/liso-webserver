\documentclass{article}
\usepackage[left=1in,top=1in,bottom=1in,right=1in]{geometry}

% \usephttps://autolab.andrew.cmu.edu/package[left=1in,top=1in,bottom=1in,right=1in]{geometry}
\usepackage{listings}
\usepackage{graphicx}
\usepackage{hyperref}
\usepackage{booktabs}
\usepackage{multicol}
\usepackage{array}

% \usepackage{breakurl}
\sloppy


\newenvironment{packed_itemize}{
\begin{list}{\labelitemi}{\leftmargin=2em}
\vspace{-6pt}
 \setlength{\itemsep}{0pt}
 \setlength{\parskip}{0pt}
 \setlength{\parsep}{0pt}
}{\end{list}}

\begin{document}

\title{	15-441/641: Computer Networks\\
Project 1: A Web Server Called Liso\\
}

\author{TAs: x <x@andrew.cmu.edu> \\
y <y@andrew.cmu.edu>}
\date{\today}

\maketitle


\section{Introduction}
In this class you wil learn about Computer Networks, from bits on a wire or radio (`the bottom') all the way up to the design of applications that run over networks (`the top)'. 
In this project, you will start from the top with an application we are all familiar with: a web server.

You will use the Berkeley Sockets API to write a web
server using a subset of the HyperText Transport Protocol (HTTP) $1.1$ ---RFC 2616~\cite{httprfc}.  Your web server will also implement
HyperText Transport Protocol Secure (HTTPS) via Transport Layer Security (TLS)
as described in RFC 2818~\cite{httpsrfc}.  Students enrolled under the 15-641 course number will
implement the Common Gateway Interface (CGI) as described in RFC
3875~\cite{cgirfc}.  Reading an RFC is quite different from reading a news article, mystery novel, or even  technical paper.
Appendix B.2 has some tips read and use and RFC efficiently.

RFCs are one of many ways the Internet declares `standards:' agreed upon algorithms, wire formats, and protocols for interoperability between different implementations of systems on a network.
Without standards, software from one company would not be able to talk to software from another company and we would not have things like e-mail, the Web, or even the ability to route traffic between different companies or countries.
Because your web server is compatible with these standards, you will, by the end of this project, be able to browse the content on your web server using standard browser like Chrome or Firefox.
 
\vspace{5pt}

\noindent With this project, you will gain experience in:
\begin{itemize}
  \item ... building non-trivial computer systems in a low-level language (C).
  \item ... complying with real industry standards as all developers of Internet services do.
  \item ... reasoning about the many tasks that application developers rely on the Internet to do for them.
\end{itemize}

\noindent To guide your development process, we have divided this project into three key checkpoints.


\begin{center}
\vspace{10pt}
\begin{tabular}{cp{2.5in}cccc}
  {\bf CP}&{\bf Goal}&\multicolumn{2}{c}{{\bf \% P1 Grade}}&\multicolumn{2}{c}{{\bf Deadline}}\\
  & & 441 & 641 & 441 & 641 \\

  \hline
  
  \addlinespace[5pt]
  1&Determine whether an HTTP 1.1 request is valid or invalid. & 25\% & 15\% & Sep 10, 2019 & Sep 6, 2019 \\
  
  \addlinespace[5pt]
  2&Respond correctly to HTTP 1.1 HEAD and GET requests. & 75\% & 25\% & Sep 27, 2019 & Sep 20, 2019   \\
  
  \addlinespace[5pt]
  3&Support HTTPS and CGI, run as a daemon, and complete all previous tasks. & - & 60\% & - & Sep 27, 2019\\
\end{tabular}
\end{center}

\noindent At each checkpoint (CP), we will assess your server to verify that it implements all of the goals for the current checkpoint 
and we will rerun tests from the previous checkpoint(s) as well. 
This means that if you fix a bug you had in an earlier checkpoint, you will get partial credit in a later checkpoint.  Of course, if you 
break a feature that was originally implemented correctly, you will lose some points, so you should return your tests from earlier 
checkpoints before you submit a later checkpoint.  This is consistent with common industry practice.
%   For this reason, the last checkpoint has the most weight assigned to it -- we will not only test HTTPS and CGI, but we will also test HEAD/GET requests (from CP 2) and error responses (from CP 1).

\newpage
\section{The Liso Server}
\label{sec:liso}

In this section, we give an overview of the the complete requirements for the Liso server: what you will turn in at the final checkpoint. 
In the next three sections (\S\ref{sec:cp1}--\ref{sec:cp3}), we break the project into a development strategy with intermediate 
checkpoints.  The starting point for your server is framework code that we will provide on the course website and in Autolab. 


\subsection{Supporting HTTP 1.1}
You will find in RFC 2616 that there are many HTTP methods: commands that a client sends to a server.
Liso will only support three methods:

\begin{itemize}
\item \textbf{GET} -- requests a specified resource; it should not have any
                       significance other than retrieval
\item \textbf{HEAD} -- asks for an identical response as GET, without the actual
                       body---no bytes from the requested resource
\item \textbf{POST} -- submit data to be processed to an identified resource;
                       the data is in the body of this request; side-effects
                       expected
\end{itemize}

\noindent{\bf Error Messages:}
For all other commands, your server must return ``501 Method Unimplemented.'' 
If
you are unable to implement one of the above commands (perhaps you ran out of
time), your server must return the error response ``501 Method Unimplemented,''
rather than failing silently (or not so silently).  

\vspace{5pt}

\noindent{\bf Robustness:} As a public server, your implementation should be
robust to client errors. 
Even if the client sends malformed inputs or breaks off the connection mid-request, your server should never crash.
For example, your server must not
overflow any buffers when a malicious client sends a message that is ``too long." 
The starter code we will provide you is not robust to malformed requests (e.g, it cannot handle requests which do not have proper [CR][LF] line endings) and so you will have to extend it.

\vspace{5pt}

\noindent{\bf Multiple Requests:} During a given connection, a client may send multiple HEAD/GET/POST requests.
The client may even send multiple requests back-to-back, without even waiting for a response.
This is called `HTTP pipelining'~\cite{pipelining}. Your server must support multiple requests in the same connection, and it must support HTTP pipelining.

\subsection{Supporting Many Clients}
Your server should be able to support multiple clients concurrently. The only
limit to the number of concurrent clients should be the number of available
file descriptors in the operating system (the min of ulimit -n and
FD\_SETSIZE---both typically 1024 but it may be higher or lower).  


While the server is waiting for a client to
send the next command, it should be able to handle inputs from other clients.
Clients may `stall' (send half of a request and then stall before sending the rest) or cause errors; these problems should not harm other concurrent users.
For example, if a client only sends half of a request and stalls, your server should move on to serving another client.
In general, concurrency can be achieved using either \texttt{select()} or
multiple threads. However, in this project, you \textbf{must implement your
server using \texttt{select()} to support concurrent connections}. Threads are
\textbf{NOT} permitted at all for the project. 

\subsection{HTTPS and CGI}
Students enrolled in 15-641 will extend your basic Liso server with support for HTTPS (which encrypts connections to your server) and CGI (which allows your server to support interactive programs) in Checkpoint 3.

\vspace{5pt}

\noindent{\bf HTTPS Support:}
Your server will use the OpenSSL library for HTTPS support.  Specially, you will wrap communication calls with with SSL wrapping functions that will encrypt data sent over the channel and authenticate the server based on a certificate.  
In order to test HTTPS, you will need to get a certificate that your HTTPS-enabled server can use to authenticate itself to clients.  

\noindent{\bf CGI Support:} The Common Gateway Interface (CGI) provides a standard interface that allows a web server to call other processes or servers.  Web servers use this typically to generate dynamic content, i.e., the content that is generated is based on input provided by the client (using POST) or other client-specific information. 
Your server will support CGI requests and will provide a built in `ASCII art generator' as a demo application.
When a client submits a POST to the ASCII art generator, the server will generate a page that spells out client-specified phrase in ASCII art.

\vspace{5pt}

\noindent More details on HTTPS and CGI are in \S\ref{sec:cp3}.
\subsection{Command Line Arguments}
\noindent\textbf{Liso will always have 8 arguments. You may not modify Liso to, e.g., to take a different number of arguments or to reorder the arguments. If you do not need one of the arguments in the earlier checkpoints, you may simply ignore it in your code.}\\

\textbf{\emph{usage:}} \emph{./lisod $<$HTTP port$>$ $<$HTTPS port$>$
                             $<$log file$>$ $<$lock file$>$ $<$www folder$>$
                             $<$CGI script path$>$ $<$private key file$>$ $<$certificate file$>$}

\begin{description}
	\item[\textnormal{\emph{HTTP port}}] -- the port for the HTTP (or echo) server
                                          to listen on

	\item[\textnormal{\emph{HTTPS port}}] -- the port for the HTTPS server to
                                           listen on

	\item[\textnormal{\emph{log file}}] -- file to send log messages to
                                         (debug, info, error)

	\item[\textnormal{\emph{lock file}}] -- file to lock on when becoming a daemon
                                          process

	\item[\textnormal{\emph{www folder}}] -- folder containing a tree to serve as
                                           the root of a website

	\item[\textnormal{\emph{CGI script name (or folder)}}] --
										   for this project, this is a file that should be
										   a script where you redirect all
										   /cgi/* URIs. In the real world, this
										   would likely be a directory of
										   executable programs.

	\item[\textnormal{\emph{private key file}}] -- private key file path

	\item[\textnormal{\emph{certificate file}}] -- certificate file path
\end{description}


\subsection{Tools, Skills, and Grading}
The projects in this course are significant larger and more complex
than the 15-213/513 projects.  They are also more open-ended, in the
sense that we only specify \emph{what} the system must do, but not
\emph{how} you do it.  For example, a web server needs to perform several 
functions, e.g., managing sessions, using the file system, generating
responses including error messages, etc.  You are responsible for the 
design: what modules you have, the data structures they use, and how they
interaction.

You will also have to strengthen your programming skills.  A big part
of that is making good use of of a variety of tools, such as {\tt
gdb}, {\tt Valgrind}, and others (see Section~\ref{sec:impl}.  We will
review these tools in recitations sessions early on in the semester.
One important skill is debugging.  You should first try to debug code
yourself (using the above tools) but if you have problems, you 
use the office hours of not only the lead TA but also the
other TAs to get help.  Note that the TAs are not allowed to debug
your code for you. Specifically, they will only look at your code for
a limited time (up to 10 minutes; leaving them time to help other
students) and they will not modify your code (they cannot touch the
keyboard).

You will also have to develop your own test suite to test and help in
debugging your code.  While we will give you some tests for each
checkpoint, these tests will only test some of the features of your
implementation and they are only a subset of the tests we will use for
grading.  You are responsible for writing additional tests so you
cover all the features and requirements listed in this handout.




\newpage
\section{Checkpoint 1: Parse and Identify Valid HTTP Requests}

In this checkpoint, you will parse requests from a client like Opera, Firefox, or Chrome.
Instead of responding to the requests, your Liso server will simply try to identify whether to request is a valid HEAD, GET, or POST request.
If it is a valid HEAD, GET, or POST request, the server will `echo' the request back to the client.
Otherwise, it will return an HTTP response with 400 as the error code.

\vspace{5pt}

\noindent Your server MUST:
\begin{itemize}
  \item Correctly identify whether an HTTP HEAD, GET, or POST request is correctly formed  \item Handle multiple clients sending HTTP requests at the same time.
  \item Use {\tt select()} to accept incoming data.
  \item Use lex and yacc to parse the requests.
\end{itemize}

\subsection{Getting Started}
\begin{enumerate}
  \item Unpack the starter code and create a git repo (\texttt{tar -zxvf Project1\_starter.tar.gz; cd 15-441-project-1; git init}).

  \item Create a \texttt{select()}-based echo server handling multiple clients at once, building on the  starter code provided on autolab and the course website.  

  \item Specifically, your server must have the capability to parse HTTP 1.1 requests and classify them as ``good'' or ``bad'' based on the provided RFC~\cite{httprfc}. For all ``good'' requests, you will simply echo back the original request. For all ``bad'' requests, you will return an HTTP response with 400 as the error code.

  \item Test using our provided cp1\_checker.py test script (read that script and understand it too.)

  \item Finally, hand-in your
submission by the deadline and include all needed files as outlined in
\S\ref{sec:handin}.\\
\end{enumerate}

\subsection{Tips \& FAQ}
    \begin{itemize}
      \item To test `Valid' requests, we will specifically be using requests as formatted by Opera, Firefox, Chrome, or Apache Bench. In industry folks test web services on a tens or even hundreds of combinations of browsers and environments... we'll stick to just these four for this project.
      \item Note that Apache Bench expects a Content-Length: field on every reply -- even error messages.
      \item If the request is validly formed but not a HEAD, GET, or POST you may either reply with a 400 or a 501 error code for this checkpoint (in CP2 you must reply with a 501).
    \end{itemize}

\newpage
\subsection{Grading}
%
The breakdown of grading for Checkpoint 1 is below.
%
\begin{center}
  \begin{tabular}{>{\centering\arraybackslash}m{1in}>{\centering\arraybackslash}m{1in}p{3in}}
  {\bf Task}&{\bf Weight}&{\bf Subcriteria}\\
  \hline
  \addlinespace[5pt]
    Format&20\%&\vspace{-10pt} 
                              {\it Assigned by human grader:}
                              \begin{packed_itemize}
                                \item Correct turnin -- Makefile, compilation, properly tagged repo, TA's don't have to edit or search for files (10\%)
                                \item Code style (5\%)
                                \item Code commenting (5\%)
                              \end{packed_itemize}\\

  \hline
  \addlinespace[5pt]

    Basics&10\%&\vspace{-10pt} 
                              {\it Assigned in AutoLab:}
                              \begin{packed_itemize}
                                \item Accepts requests
                                \item Provides a response
                                \item Does not crash on receiving valid or invalid requests
                                \item Accepts/Replies to requests from multiple clients at the same time
                              \end{packed_itemize}\\

  \hline
  \addlinespace[5pt]

  Valid Request Handling&25\%&\vspace{-10pt} 
                              {\it Assigned in AutoLab:}
                              \begin{packed_itemize}
                                \item Echoes to valid requests from Opera, Chrome, Firefox, or Apache Bench.
                              \end{packed_itemize}\\

  \hline
  \addlinespace[5pt]

  Invalid Request Handling&45\%&\vspace{-10pt} 
                              {\it Assigned in AutoLab:}
                              \begin{packed_itemize}
\item Provides error code 400 to requests formatted improperly according to RFC 2616.
                              \end{packed_itemize}\\

\end{tabular}
\end{center}


\noindent Checkpoint 1 is worth 15\% of P1, which is worth 15\% of your final grade. Hence, in completing Checkpoint 1 with a perfect score you would earn 2.25\% of the points available in this class.



\newpage
\section{Checkpoint 2}
\label{sec:cp2}
In this checkpoint, your web server will reply to client requests as a realy HTTP web server for static content. You will be able to browse to your server using a real web browser!

\vspace{5pt}

\noindent Your server MUST:
\begin{itemize}
  \item Respond to properly formatted HTTP HEAD and GET requests. You do not need to respond to POST requests yet (reply to these with error 501).
    \item Support three HTTP 1.1 error codes: 400, 501, and 505. 501 is for unsupported methods; 505 is for bad version numbers. Everything else can be handled with an 400.
    \item Handle concurrent connections using select().
    \item Handle pipelined requests.
\end{itemize}

\subsection{Getting Started}
\begin{enumerate}
\item Begin with your repository for Checkpoint 2, using the result of Checkpoint 1 as a starting point.
\item You are highly encouraged to create a simplified logging module for your project that writes out formatted logs to the log file specified on the command-line; this will help you debug and trace requests that come to your system. {\it Nonetheless we will not require this in grading.}
\item Enhance your server to respond properly to any HTTP 1.1 request and implement persistent connections with HEAD and GET as defined in RFC 2616. At this point as we don't have CGI we will not check responses for POST requests.
\item Your server will need to handle lots of concurrent and pipelined requests: make sure you test with many simultaneous connections.
\item The server should also handle errors in a sane way. {\bf It should never completely crash} (make it as robust as possible). In testing, try to crash your server by sending malformed and strange requests and fix your server to prevent these crashes.
\item Double check that your server sends the appropriate error messages to malformed requests.
\item Submission is the same as Checkpoint 1. Tag and upload your repo in a tarball to the corresponding lab on Autolab.
\end{enumerate}


\subsection{Tips and FAQ}
\begin{enumerate}
    \item Do we have to support Chunking?\\
    \textbf{No}
    %\item For POST, if a Content-Length header is not provided in the request, is it fine to respond with a 411 as specified in the RFC?\\
    %\textbf{Yes}, this is both possible and desired because it is a valid response. It may also simplify the design of the server.
    \item Do we have to support "Conditional" GETs?\\
    \textbf{No.}
    \item Can I use a hash table library written by another person?\\
    \textbf{No}, for this project implement your own if you really want it. You won't have to track every header, only the important ones for basic compliance.
    \item Should we expect HTTP/1.1 requests to fit in one buffer?\\
    \textbf{No}, do not assume that they always fit inside one buffer. Be prepared to parse across buffer boundaries.
    \item Can I assume that the request always has the Content-length field?\\
    \textbf{Yes}, assume requests to your server have Content-length if applicable. If it is missing, return a 411 response.
    \item Since the server will serve static files, are we allowed to use http://svn.apache.org/repos/asf/httpd/httpd/trunk/docs/conf/mime.types\\
    \textbf{Yes}, You are allowed to use that or a simplified version of it.No need to support all well-known MIME-types, just the most common ones: text/html, text/css, image/png, image/jpeg, image/gif and maybe a few others up to your discretion.
    \item For last-modified field, is there a C library that we can read file
     metadata  such as this? Or is our web server supposed to handle that
     manually. (i.e. stamping each and every file in the www root with a
     last-modified stamp and manually updating that stamp every time a file is
     changed....).\\
    \textbf{stat()} is a system call to check for metadata on a file. 
    \item Are we allowed to reject requests with headers beyond a maximum size?\\
    \textbf{Yes}, you may reject any header line larger than 8192 bytes. Note, that this is different than a Content Length of greater than 8192. Additionally, you must find and parse the next request properly for pipelining purposes if you do reject the request.
    \item Should we also handle request made up with /n instead of /r/n?\\
    Some web servers do this and it is nice for telnet testing. However, Liso does not have this as a requirement -- you do not have to do this.
\end{enumerate}

\subsection{Grading}

The breakdown of grading for Checkpoint 2 is below.
%
\begin{center}
  \begin{tabular}{>{\centering\arraybackslash}m{1in}>{\centering\arraybackslash}m{1in}p{3in}}
  {\bf Task}&{\bf Weight}&{\bf Subcriteria}\\
  \hline
  \addlinespace[5pt]
    Format&10\%&\vspace{-10pt} 
                              {\it Assigned by human grader:}
                              \begin{packed_itemize}
                                \item Correct turnin -- Makefile, compilation, properly tagged repo, TA's don't have to edit or search for files (10\%)
                                \item Code style (5\%)
                                \item Code commenting (5\%)
                              \end{packed_itemize}\\

  \hline
  \addlinespace[5pt]

    Error Handling&15\%&\vspace{-10pt} 
                              {\it Assigned in AutoLab:}
                              \begin{packed_itemize}
                                \item Responds to wrong version with 505
                                \item Responds to unsupported method with 501
                                \item Responds to other problems / invalid requests with 400.
                              \end{packed_itemize}\\

  \hline
  \addlinespace[5pt]

  Request Handling&45\%&\vspace{-10pt} 
                              {\it Assigned in AutoLab:}
                              \begin{packed_itemize}
                                \item Replies to correctly formatted HEAD requests.
                                \item Replies to correctly formatted GET requests.
                                \item Accepts and responds to pipelined requests.
                              \end{packed_itemize}\\

  \hline
  \addlinespace[5pt]

  System Design&30\%&\vspace{-10pt} 
                              {\it Assigned in AutoLab:}
                              \begin{packed_itemize}
                                \item Never crashes.
                                \item Uses select() (no multithreading).
                                \item Handles hundreds of concurrent connections when tested with Apache Bench.
                              \end{packed_itemize}\\

\end{tabular}
\end{center}


\noindent Checkpoint 2 is worth 25\% of P1, which is worth 15\% of your final grade. Hence, in completing Checkpoint 2 with a perfect score you would earn 3.75\% of the points available in the final grade.



\newpage
\section{Checkpoint 3}
\label{sec:cp3}
In this checkpoint, students enrolled in 15-641 will add support for
CGI requests for HTTPS, which uses encryption to keep connection
content private from Internet eavesdroppers.  In addition, you will
`daemonize' your server to allow it to run as a persistent background
task, and you will double-check all of the requirements from
Checkpoints 1 and 2 for a final grade.

\vspace{5pt}
\noindent Your server MUST:
\begin{itemize}
  \item Support the CGI standard as specified in RFC 3875.
  \item Support HTTPS via TLS as specified in RFC 2818.
  \item Run as a daemonized, background process.
  \item {\bf Remember: Continue to support regular HTTP requests on port 80, as in CP2.}
\end{itemize}

\subsection{Getting Started}
We suggest you work on this checkpoint in two phases. In one phase, implement CGI. In another phase, add support for HTTPS. You can do these steps in either order -- just don't try to do both at the same time!

\subsubsection{Implementing CGI}

  Most of the instructions for this checkpoint are listed below. But, we recommend that you `skim' the CGI RFC~\cite{cgirfc} and the URI RFC~\cite{urirfc} to get the big picture before diving in to your implementation.

The remainder of this section explains the steps involved in completing CP3.  A high-level overview of
CGI and HTTPS was presented in the web lecture in the beginning of the semester.

\begin{enumerate}
  
  \item Create support for the CGI variables listed in \S\ref{sec:cgi-vars}. Not all variables may be used in the ASCII Art sample, but we will have additional checks to make sure that you do set these variables in case other CGI scripts are used. Using a python CGI script that echos the environment variables may be useful for your debugging.
\item Implement your CGI module. Any URI starting with ``/cgi/'' will be handled by a single command-line specified executable via a CGI interface coded by you. We have also provided a CGI runner in C. 
\begin{enumerate}
    \item CGI URI's may accept GETs, POSTs, and HEADs; your job is not to decide this, just pass along information to the program being called
    \item You need to pipe stdin, pipe stdout, fork(), setup environment variables per the CGI specification, and execve() the executable (it should be executable) Note: Watch the piped fd's in the parent process using your select() loop. Just add them to the appropriate select() sets and treat them like sockets, except you have to pipe them further to specific sockets.
    \item Pass any message body (especially for POSTs) via stdin to the CGI executable
    \item Receive any response over stdout until the process dies (monitor process status), or there is nothing more to read or a broken pipe is encountered
    \item If the CGI program fails in any way, return a 500 response to the client, otherwise send all bytes from the stdout of the spawned process to the requesting client.
    \item The CGI application will produce headers and message body as it sees fit, you do not need to modify or inspect these bytes at all.
\end{enumerate}
\end{enumerate}

\subsubsection{Implementing HTTPS}

\begin{enumerate}
\item Create a DNS hostname for yourself with a free account at No-IP \cite{WWW:no-ip} (or use a domain name you already have...)
\item In order to test your HTTPS implementation, we need to monitor your (encrypted!) HTTPS traffic.  To allow this, add the 15-441 Carnegie Mellon University Root CA to your browser (import certificate, usually somewhere in preferences)
\begin{enumerate}
    \item Now we can man-in-the-middle your HTTPS :-) Being course staff has perks!
    \item But just trust us till this part is over...or make your own \textbf{\href{http://www.g-loaded.eu/2005/11/10/be-your-own-ca/}{CA}}.
    \item Really though, this is the part of the course where you need to Reflect on Trusting Trust.
\end{enumerate}
\item Obtain your own private key and public certificate from the 15-441 CMU CA.  You will need this when you run your HTTPS-enabled web server.
\item Implement SSL support - we have provided you a sample C server in the Autolab Handout. 
\begin{enumerate}
    \item Use the OpenSSL library. \cite{www:openssl}
    \item Create a second server listening socket in addition to the first one. Use the passed in SSL port from the commandline arguments.
    \item Add this socket to the select() loop just like your normal HTTP server socket.
    \item Whenever you accept connections, wrap them with the SSL wrapping functions.
    \item Use the special read() and write() SSL functions to read and write to these special connected clients
    \item If you setup your browser, you may now verify that connections to your webserver use TLSv1.0; inspect the ciphers, message authentication hash scheme, and key exchange methods used by your server.
\end{enumerate}
\item Implement daemonization - we have provided you a sample of daemonizing code.
\item Submission is the same as Checkpoint 1. Tag and upload your repo in a tarball to the corresponding lab on Autolab.
\end{enumerate}

\subsection{Grading}
%
The breakdown of grading for Checkpoint 3 is below.
%
\begin{center}
  \begin{tabular}{>{\centering\arraybackslash}m{1in}>{\centering\arraybackslash}m{1in}p{3in}}
  {\bf Task}&{\bf Weight}&{\bf Subcriteria}\\
  \hline
  \addlinespace[5pt]
    Format&10\%&\vspace{-10pt} 
                              {\it Assigned by human grader:}
                              \begin{packed_itemize}
                                \item Correct turnin -- Makefile, compilation, properly tagged repo, TA's don't have to edit or search for files (10\%)
                                \item Code style (5\%)
                                \item Code commenting (5\%)
                              \end{packed_itemize}\\

  \hline
  \addlinespace[5pt]

    CGI&30\%&\vspace{-10pt} 
                              {\it Assigned in AutoLab:}
                              \begin{packed_itemize}
                                \item Supports all CGI variables
                                \item Runs provided ASCII Art CGI script
                                \item Accepts/Replies to CGI requests from multiple clients at the same time over HTTP.
                              \end{packed_itemize}\\

  \hline
  \addlinespace[5pt]

  HTTPS&20\%&\vspace{-10pt} 
                              {\it Assigned in AutoLab:}
                              \begin{packed_itemize}
                                \item Presents a certificate and initiates a TLS connection.
                                \item GET, HEAD, and POST requests all work through the HTTPS connection.
                                \item Multiple clients can connect at the same time over HTTPS.
                              \end{packed_itemize}\\

  \hline
  \addlinespace[5pt]

  Basic HTTP&40\%&\vspace{-10pt} 
                              {\it Assigned in AutoLab:}
                              \begin{packed_itemize}
                                \item All criteria from CP2.
                              \end{packed_itemize}\\

\end{tabular}
\end{center}


\noindent Checkpoint 3 is worth 60\% of P1, which is worth 15\% of your final grade. Hence, in completing Checkpoint 3 with a perfect score you would earn 9\% of the points available in this class.



\newpage
\section{Implementation and Submission}
\label{sec:impl}
\section{Implementation and Submission}
This is a solo project: you \emph{must} implement and submit your own code.  
Source materials for this project may be found on Autolab. 

\subsection{Coding and Version Control}
Your server must be written in the C programming language. You are not allowed
to use any custom socket classes or libraries, only the standard socket library
and the provided library functions. You \textbf{may not} use the csapp wrapper
library from 15-213/15-513 or libpthread for threading. We disallow csapp.c for two
reasons:
first, to ensure that you understand the raw standard BSD sockets API,
and, second, because csapp.c's wrapper functions are not suitable for robust
servers. Temporary system call failures (e.g., EINTR) in functions such as
\texttt{select()} could cause the server to abort and utility functions like
\texttt{rio\_readlineb} are not designed for nonblocking code.

That said, we encourage the use of \emph{anything} for testing.  Use
Wireshark~\cite{wireshark}, use telnet, use real web browsers, use Python to
script tests---for testing, the sky is the limit.

All of your project files and submissions \textbf{must} be stored in a git repository.
%% repository: \url{http://sourcery.cmcl.cs.cmu.edu/indefero}

%%\item All of your project files and submissions \textbf{must} be stored on
%%  AFS in the 441 course directory (\texttt{/afs/andrew/course/15/441-641/}) as a
%%  git-backed repository. You should have a subdirectory in that folder named the
%%  same as your Andrew ID where you must store your git repo (i.e., if your Andrew
%%  ID is \emph{mmukerje} then store your git repo at
%%  \texttt{/afs/andrew/course/15/441-641/mmukerje/}).

You will submit your code as a tarball named \texttt{\textit{<andrewID>}.tar}. Untarring this file should give us a directory named \texttt{15-441-project-1} which should contain the git repository as well as the code. You will submit this tarball using Autolab ( \url{https://autolab.andrew.cmu.edu} ). If you still can't login with your andrew ID by the end of January 22nd, let us know ASAP.\\\\
Checkpoints are designed to ensure that you keep tabs on your progress and are a great guideline to help you complete your project on time. Please note that  Checkpoint 1 is fairly straightforward and a small part of the total work. It helps you to familiarize yourself with the basics of socket programming (specifically how to use \texttt{select()} system call) and  will require some amount of RFC interpretation for implementing a basic HTTP 1.1 parser. The rest of the checkpoints are progressively harder and will build on the previous checkpoints.


\subsection{Compiling}
You are responsible for making sure your code compiles and runs correctly on
the Andrew x86 machines running Linux (i.e., \texttt{linux.andrew.cmu.edu} /
\texttt{unix.andrew.cmu.edu}). We recommend using \texttt{gcc} to compile your
program and \texttt{gdb} to debug it. You should use the \texttt{-Wall} and
\texttt{-Werror} flags when compiling to generate full warnings and to help
debug. Other tools available on the Andrew unix machines that are suggested are
ElectricFence~\cite{WWW:efence} (link with \texttt{-lefence}) and
Valgrind~\cite{WWW:Valgrind}---use this with full leak checking to ensure you
have no memory leaks.  For this project, \textbf{you will also be responsible
for turning in a GNU Make compatible Makefile}. See the GNU make
manual\cite{Manual:Make} for details. When we run \texttt{make} we should end
up with the Liso web server binary \textbf{\texttt{lisod}}.
\subsection{Running}
%This is how we will start your implementation of Liso:\\
%\emph{
%\indent./lisod 8080 4443 /tmp/lisod.log /tmp/liso.lock /tmp/www /tmp/cgi/cgi_script.py /tmp/priv.key /tmp/cert.crt \\
%}
The Liso server will be passed the ports to run on, what log file to use, what
lock file to use when daemonizing, folders to serve static data from as well
as CGI applications, and TLS private/public key pairs.

Not all of these options need to be functional at each stage of development.
Only a HTTP port is needed for the first checkpoint when implementing an echo server
using \texttt{select()}.

\subsection{Framework Code} 
We will provide you with framework code that will, for example, help in forking
a process for proper CGI handling and setting up the environment, parse
commandline arguments (and sanity check them) and daemonize a process.

\textbf{DISCLAIMER:} We reserve the right to change the support code as the
project progresses to fix bugs and to introduce new features that will help you
debug your code. You are responsible for checking Piazza to stay
up-to-date on these changes. We will assume that all students in the class will
read and be aware of any information posted to Piazza.

\subsection{HTTP Packet Parsing}
Historical evidence suggests that most students spend considerable
amount of time writing correct parsers. While parsing packets using
C's string manipulation functions may well be an essential skill to
have, it might get insanely tedious. We want you to spend time on
other more important programming aspects such as socket programming,
handling race conditions and memory leaks. For this reason, we require you to use Lex and Yacc for parsing packets. We will also provide you with a basic HTTP parser written in lex and yacc. More about parsing
using Lex and Yacc will be covered in recitations. So, stay tuned!


\subsection{Hand-In}
\label{sec:handin}
Handing in code for checkpoints and the final submission deadline will be done
through Autolab (\url{https://autolab.andrew.cmu.edu}).  You are supposed to upload the tarball file for each checkpoint into the corresponding assessment on our course page of Autolab website.

\subsection{Work with git}
You are supposed to create your git repo on your local machine or on a \textbf{private} git repo hosted
online as part of
Checkpoint 1. Every checkpoint will be a git tag in this repo. To create a tag, run
\[
\texttt{git tag -a checkpoint-\textit{<num>} -m \textit{<message>} [\textit{<commit hash>}]}
\]
with appropriate checkpoint number and custom message filled in. (Put whatever
you like for the message --- git won't let you omit it.) The optional commit
hash can be used to specify a particular commit for the tag; if you omit it,
the current commit is used. If you choose to clone your repository onto your
local machine for development, be sure to use \texttt{git push --tags} to sync
your work back to git server; the standard \texttt{git push} doesn't send tags.

\subsection{Upload your code}
To submit your code, make a tarball file of you repo after you tag it. Then login to autolab website, choose \texttt{“15-441: Computer Networks (s19)” -> “project1cp\textit{<N>}”} and then upload your tarball. The grader should be finished in less than a minute but may take longer depending on system load. When it is done, your score will be shown. Only the latest score will be used.\\


Untarring the tarball should give us a directory named \texttt{15-441-project-1} which contains a valid git repo with tags. Your repo should contain the minimum following files:
\begin{itemize}
	\item \textbf{Makefile} -- Make sure all the variables and paths are set
      correctly such that your program compiles in the handin directory---not
      just a local machine or account. The Makefile should, by default, always
      build an executable named lisod.
	\item \textbf{All of your source code} -- (files ending with .c, .h,
      etc. only, no .o files and no executables)
	\item \textbf{readme.txt} -- File containing a brief description of your source tree organization.
	
	\item \textbf{tests.txt} -- File containing documentation of your test cases
      and any known issues you have.
%    \item \textbf{replay.test} -- File containing input to send to the server
%      testing the implementation
%    \item \textbf{replay.out} -- File containing expected server output matching
%      input given in replay.test
    %\item %\textbf{vulnerabilities.txt} -- File containing documentation of at
    %  least one vulnerability you identify at each stage.
\end{itemize}

	Late submissions will be handled according to the policy given in the
        course syllabus.

%\newpage




\appendix

\newpage
\section{Required CGI Variables}
\label{sec:cgi-vars}

We will test for the following CGI variables only.
\begin{enumerate}
\item CONTENT\_LENGTH -- taken directly from request
\item CONTENT\_TYPE -- taken directly from request
\item GATEWAY\_INTERFACE -- "CGI/1.1"
\item PATH\_INFO -- $<path>$ component of URI
\item QUERY\_STRING -- parsed from URI as everything after ``?''
\item REMOTE\_ADDR -- taken when accept() call is made
\item REQUEST\_METHOD -- taken directly from request
\item REQUEST\_URI -- taken directly from request
\item SCRIPT\_NAME -- hard-coded/configured application name (virtual path) 
\item SERVER\_PORT -- as configured from command line (HTTP or HTTPS port depending)
\item SERVER\_PROTOCOL -- "HTTP/1.1"
\item SERVER\_SOFTWARE -- "Liso/1.0"
\item HTTP\_ACCEPT -- taken directly from request
\item HTTP\_REFERER -- taken directly from request
\item HTTP\_ACCEPT\_ENCODING -- taken directly from request
\item HTTP\_ACCEPT\_LANGUAGE -- taken directly from request
\item HTTP\_ACCEPT\_CHARSET -- taken directly from request
\item HTTP\_HOST -- taken directly from request
\item HTTP\_COOKIE -- taken directly from request
\item HTTP\_USER\_AGENT -- taken directly from request
\item HTTP\_CONNECTION -- taken directly from request
\item HTTP\_HOST -- taken directly from request
\end{enumerate}


\newpage
\section{Tips}

\section{Getting Started}
This section gives suggestions for how to approach the project. Naturally, other
approaches are possible, and you are free to use them.

\begin{itemize}
	\item \textbf{Start early!} The hardest part of getting started tends to be
	  getting started. Remember the 90-90 rule: the first 90\% of the job takes
	  90\% of the time; the remaining 10\% takes the other 90\% of the time.
	  Starting early gives you time to ask questions. For clarifications on
	  this assignment, post to Piazza and read project updates on the course
	  web page.  Talk to your classmates. While you need to write your own
	  original program, we expect conversation with other people facing the
	  same challenges to be very useful. Come to office hours. The course staff
	  is here to help you.

	\item Read the RFCs selectively. RFCs are written in a style that you may
      find unfamiliar. However, it is wise for you to become familiar with it,
      as it is similar to the styles of many standards organizations. We don't
      expect you to read every page of the RFC, especially since you are only
      implementing a small subset of the full protocol, but you may well need to
      re-read critical sections a few times for the meaning to sink in.

	\item Begin by taking a cursory first pass over the RFCs. Do not focus on
      the details; just try to get a sense of how they work at a high
      level. Understand the role of the server. Understand what error conditions
      are possible, and how they are used. You may want to print the RFCs, and
      mark them up to indicate which parts are important for this project, and
      which parts are not needed. You may need to re-read these sections several
      times.

	\item Next, take a second pass over the RFCs. You will want to read all of
      them together.  Again, do not focus on the details; just try to understand
      the requests and responses at a high level. As before, you may want to
      mark up a printed copy to indicate which parts of the RFCs are important
      for the project, and which parts are not needed.

	\item Now, go back and read with an eye toward implementation. Mark the
      parts which contain details that you will need to write your server.
      Start thinking about the data structures (input and output buffers, etc.)
      your server will need to maintain. What information needs to be stored
      about each client while servicing requests (maybe an HTTP 1.1 finite state
      machine per client, etc.)?

	\item Get started with a simple server that accepts connections from
	  multiple clients. It should take any message sent by any client, and
	  ``echo'' that message back to its sender.  This server will not be
	  compatible with HTTP clients, but the code you write for it will be
	  useful for your final server. Writing this simpler server will let you
	  focus on the socket programming aspects of a server without worrying
	  about the details of the protocols. Test this simple server with the provided Python script in Checkpoint 1.

    \item Next, enhance the starter code we have provided for HTTP parsing. Apply all the RFC knowledge you have gathered from previous steps and try to convert them into rules. After you combine this parser along with the echo server, you should be ready for Checkpoint 1. 
    
	\item At this point, you are ready to write a standalone HTTP 1.1
      server. But do not try to write the whole server at once. Decompose the
      problem so that each piece is manageable and testable.  For each request,
      identify the different cases that your server needs to handle. Find common
      tasks among different commands and group them into procedures to avoid
      writing the same code twice. You might start by implementing the routines
      that read and parse commands. Then implement commands one by one, testing
      each with \texttt{telnet}.

	\item Thoroughly test your server. Use the provided scripts to test basic
	  functionality. For further testing, use \texttt{telnet}, a web browser,
	  or replay scripts. Learn Python from our scripts and as we go to make
	  repeatable ``regression tests''---every time you implement a new feature
	  you use regression tests to see if anything broke.

	\item Make sure to check the return code of all system calls and handle
      errors appropriately. Temporary failures (e.g., EINTR) should not cause
      your server to abort or exit in failure.  Fatal errors can be dealt with
      via a perror() call and exiting---but try to clean up open file descriptors
      and sockets nicely even when fatally exiting.

	\item Be liberal in what you accept and conservative in what you
      send~\cite{RFC:1122}.  Following this guiding principle of Internet design
      will help ensure your server works with many different and unexpected
      client behaviors.

	\item Code quality is important. Make your code modular and extensible where
      possible. You should probably invest an equal amount of time in testing
      and debugging as you do writing. Also, debug incrementally.  Write in
      small pieces and make sure they work before going on to the next piece.
      Your code should be readable and commented. Not only should your code be
      modular, extensible, readable, etc, most importantly, it should be your
      own!

    \item You may want to consider turning warnings into errors to avoid bad
	  programming style. Do this by passing \texttt{-Werror} to \texttt{gcc}
	  during compilation.

	\item If you have a question about a project handout or a technical issue,
	  there is an excellent chance that other students have the same question.
	  Please read Piazza to see if there has been traffic and consider posting
	  your questions there.

\end{itemize}

\section{Resources}
For information on network programming, the following may be helpful:

\begin{itemize}
    \item Beej's Guide~\cite{beej}
	\item Class Textbook -- Sockets, etc
	\item Class Piazza --  Announcements, clarifications, etc
	\item Class Website --  Announcements, errata, etc
	\item Computer Systems: A Programmer's Perspective (CS 15-213 text book)\cite{Book:CSAPP}
	\item BSD Sockets: A Quick And Dirty Primer\cite{BSD:Sockets}
	\item An Introductory 4.4 BSD Interprocess Communication Tutorial\cite{BSD:IPC}
	\item Unix Socket FAQ\cite{FAQ:Sockets}
	\item Sockets section of the GNU C Library manual
		\begin{itemize}
			\item Installed locally: info libc
			\item Available online: GNU C Library manual\cite{Manual:libc}
		\end{itemize}
	\item man pages
		\begin{itemize}
			\item Installed locally (e.g. man socket)
			\item Available online: the Single Unix Specification\cite{Spec:Unix}
		\end{itemize}
	\item Other Google Groups / Stackoverflow - Answers to almost anything\cite{Google:Groups}
\end{itemize}




\newpage
\section{Resources}

For information on network programming, the following may be helpful:

\begin{itemize}
    \item Beej's Guide~\cite{beej}
	\item Computer Systems: A Programmer's Perspective (CS 15-213 text book)\cite{Book:CSAPP}
	\item BSD Sockets: A Quick And Dirty Primer\cite{BSD:Sockets}
	\item An Introductory 4.4 BSD Interprocess Communication Tutorial\cite{BSD:IPC}
	\item Unix Socket FAQ\cite{FAQ:Sockets}
	\item Sockets section of the GNU C Library manual\cite{Manual:libc}
	\item man pages
		\begin{itemize}
			\item Installed locally (e.g. man socket)
			\item Available online: the Single Unix Specification\cite{Spec:Unix}
		\end{itemize}
\end{itemize}




\newpage

\begin{thebibliography}{1}
\bibitem{urirfc}
RFC 2396: \url{http://www.ietf.org/rfc/rfc2396.txt}
\bibitem{httprfc}
RFC 2616: \url{http://www.ietf.org/rfc/rfc2616.txt}
\bibitem{httpsrfc}
RFC 2818: \url{http://www.ietf.org/rfc/rfc2818.txt}
\bibitem{cgirfc}
RFC 3875: \url{http://www.ietf.org/rfc/rfc3875}
\bibitem{apache}
Apache: \url{http://httpd.apache.org/}
\bibitem{wireshark}
Wireshark: \url{http://www.wireshark.org/}
%\bibitem{flex}
%GNU Flex: \url{http://flex.sourceforge.net/manual/}
%\bibitem{bison}
%GNU Bison: \url{http://www.gnu.org/s/bison/manual/bison.html}
\bibitem{www:openssl}
Open SSL: \url{https://www.openssl.org/docs/manmaster/man3/}
\bibitem{beej}
Beej's Guide: \url{http://beej.us/guide/bgnet/output/html/singlepage/bgnet.html}
\bibitem{Manual:Make}
GNU Make Manual:  \url{http://www.gnu.org/software/make/manual/make.html}
\bibitem{RFC:1122}
RFC 1122 \url{http://www.ietf.org/rfc/rfc1122.txt}, page 11
\bibitem{WWW:efence}
ElectricFence: \url{http://perens.com/FreeSoftware/ElectricFence/}
\bibitem{WWW:common_log}
Common Log Format: \url{https://httpd.apache.org/docs/1.3/logs.html#common}
\bibitem{WWW:no-ip}
No-IP: \url{https://www.noip.com/free}
\bibitem{WWW:Valgrind}
Valgrind: \url{http://valgrind.org/}
\bibitem{Book:CSAPP}
CSAPP: \url{http://csapp.cs.cmu.edu}
\bibitem{BSD:Sockets}
\url{http://www.cis.temple.edu/~ingargio/old/cis307s96/readings/docs/sockets.html}
\bibitem{BSD:IPC}
\url{http://docs.freebsd.org/44doc/psd/20.ipctut/paper.pdf}
\bibitem{FAQ:Sockets}
\url{http://www.developerweb.net/forum/forumdisplay.php?s=f47b63594e6b831233c4b8ebaf10a614&f=70}
\bibitem{Manual:libc}
\url{http://www.gnu.org/software/libc/manual/}
\bibitem{Spec:Unix}
\url{http://www.opengroup.org/onlinepubs/007908799/}
\bibitem{Google:Groups}
\url{http://groups.google.com}
%%%
\end{thebibliography}





\end{document}


