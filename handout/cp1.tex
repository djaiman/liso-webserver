\label{sec:cp1}
In this checkpoint, you will parse requests from a client like Opera, Firefox, or Chrome.
Instead of responding to the requests, your Liso server will simply try to identify whether to request is a valid HEAD, GET, or POST request.
If it is a valid HEAD, GET, or POST request, the server will `echo' the request back to the client.
Otherwise, it will return an HTTP response with 400 as the error code.

\vspace{5pt}

\noindent Your server MUST:
\begin{itemize}
  \item Correctly identify whether an HTTP HEAD, GET, or POST request is correctly formed  \item Handle multiple clients sending HTTP requests at the same time.
  \item Use lex and yacc to parse the requests.
  \item Use {\tt select()} to accept incoming data.
  \item Never crash for any error.
\end{itemize}

\subsection{Getting Started}
\begin{enumerate}
  \item Unpack the starter code and create a git repo (\texttt{tar -zxvf Project1\_starter.tar.gz; cd 15-441-project-1; git init}).

  \item Create a \texttt{select()}-based echo server handling multiple clients at once, building on the  starter code.  

  \item Parse HTTP 1.1 requests and classify them as ``good'' or ``bad'' based on the provided RFC~\cite{httprfc}. For all ``good'' requests, you will simply echo back the original request. For all ``bad'' requests, you will return an HTTP response with 400 as the error code.

  \item Test using our provided cp1\_checker.py test script (read that script and understand it too.) Try to `break' your server and make it crash -- and then patch the bugs you find that makes it crash.

  \item Finally, hand-in your
submission by the deadline and include all needed files as outlined in
\S\ref{sec:handin}.\\
\end{enumerate}

\subsection{Tips \& FAQ}
    \begin{itemize}
      \item To test `Valid' requests, we will specifically be using requests as formatted by Opera, Firefox, Chrome, or Apache Bench. In industry folks test web services on a tens or even hundreds of combinations of browsers and environments... we'll stick to just these four for this project.
    
      \item The RFC is long, and there are many tricky points in determining whether or not a request is `Valid'. {\it It is not cheating to discuss what the RFC says.} Clarification about what the RFC requires is absolutely okay discussion!  

      \item Note that Apache Bench expects a Content-Length: field on every reply -- even error messages.
      \item If the request is validly formed but not a HEAD, GET, or POST you may either reply with a 400 or a 501 error code for this checkpoint (in CP2 you must reply with a 501).
  
      \item Can we assume that we will only receive one request per connection?\\
        {\bf Yes}, we will not require multi-request support or pipelining until CP2.
    \end{itemize}

\newpage
\subsection{Grading}
%
The breakdown of grading for Checkpoint 1 is below.
%
\begin{center}
  \begin{tabular}{>{\centering\arraybackslash}m{1in}>{\centering\arraybackslash}m{1in}p{3in}}
  {\bf Task}&{\bf Weight}&{\bf Subcriteria}\\
  \hline
  \addlinespace[5pt]
    Format&20\%&\vspace{-10pt} 
                              {\it Assigned by human grader:}
                              \begin{packed_itemize}
                                \item Correct turnin -- Makefile, compilation, properly tagged repo, TA's don't have to edit or search for files (10\%)
                                \item Code style (5\%)
                                \item Code commenting (5\%)
                              \end{packed_itemize}\\

  \hline
  \addlinespace[5pt]

    Basics&10\%&\vspace{-10pt} 
                              {\it Assigned in AutoLab:}
                              \begin{packed_itemize}
                                \item Accepts requests
                                \item Provides a response
                                \item Does not crash on receiving valid or invalid requests
                                \item Accepts/Replies to requests from multiple clients at the same time
                              \end{packed_itemize}\\

  \hline
  \addlinespace[5pt]

  Valid Request Handling&25\%&\vspace{-10pt} 
                              {\it Assigned in AutoLab:}
                              \begin{packed_itemize}
                                \item Echoes to valid requests from Opera, Chrome, Firefox, or Apache Bench.
                              \end{packed_itemize}\\

  \hline
  \addlinespace[5pt]

  Invalid Request Handling&45\%&\vspace{-10pt} 
                              {\it Assigned in AutoLab:}
                              \begin{packed_itemize}
                                  \item Provides error code 400 to requests formatted improperly according to RFC 2616.

                                \item Does not cause problems or slow down concurrent clients just because one client breaks or hangs.
                              \end{packed_itemize}\\

\end{tabular}
\end{center}


\noindent Checkpoint 1 is worth 15\% of P1, and P1 is worth 15\% of your final grade. Hence, in completing Checkpoint 1 with a perfect score you would earn 2.25\% of the points available for the final grade.

