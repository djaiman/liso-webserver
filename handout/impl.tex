In this section we talk about your code and requirements for development.
Recall that in all checkpoints, part of your grade is based on meeting the formatting requirements. So, please read this section carefully!

\subsection{Framework Code} 
We will provide you with framework code that will, for example, help in forking
a process for proper CGI handling and setting up the environment, parse
commandline arguments (and sanity check them) and daemonize a process.
You may download this code from the course website or from Autolab.

\subsection{Coding and Compilation}
Your server must be written in the C programming language. 
You may use code from the 15-213/15-513 course, so long as you cite the borrowed
code in a reference. 
Nonetheless, we warn you that the 15-213/513 code will require modification (e.g. it can cause your server to crash/abort in certain conditions; the code is blocking and can impede your ability to handle concurrent connections)
You are not allowed
to use any other third party socket classes or libraries, only the standard socket library
and the provided library functions in the starter code 


You are responsible for making sure your code compiles and runs correctly on
the Andrew x86 machines running Linux (i.e., \texttt{linux.andrew.cmu.edu} /
\texttt{unix.andrew.cmu.edu}). We recommend using \texttt{gcc} to compile your
program and \texttt{gdb} to debug it. You should use the \texttt{-Wall} and
\texttt{-Werror} flags when compiling to generate full warnings and to help
debug. Other tools available on the Andrew unix machines that are suggested are
ElectricFence~\cite{WWW:efence} (link with \texttt{-lefence}) and
Valgrind~\cite{WWW:Valgrind}---use this with full leak checking to ensure you
have no memory leaks.  For this project, you will also be responsible
for turning in a GNU Make compatible Makefile. See the GNU make
manual\cite{Manual:Make} for details. When we run \texttt{make} we should end
up with the Liso web server binary \textbf{\texttt{lisod}}.


\subsection{Work with git}


All of your project files and submissions \textbf{must} be stored in a git repository.

You are supposed to create your git repo on your local machine or on a \textbf{private} shared repo hosted
online as part of
Checkpoint 1. 
{\bf If you use github do not make your project code public or you will find cheaters copying your code and yourself in a very uncomfortable academic misconduct meeting.}

Every checkpoint will be a git tag in your repo. To create a tag, run
\[
\texttt{git tag -a checkpoint-\textit{<num>} -m \textit{<message>} [\textit{<commit hash>}]}
\]
with appropriate checkpoint number and custom message filled in. (Put whatever
you like for the message --- git won't let you omit it.) The optional commit
hash can be used to specify a particular commit for the tag; if you omit it,
the current commit is used. 
Be sure to use \texttt{git push --tags} to sync
your work back to git server; the standard \texttt{git push} doesn't synchronize tags.

\subsection{Hand-In}
\label{sec:handin}

To submit your code, make a tarball file of you repo after you tag it. 
You will submit your code as a tarball named \texttt{\textit{<andrewID>}.tar}. Untarring this file should give us a directory named \texttt{15-441-project-1} which should contain the git repository as well as the code. You will submit this tarball using Autolab ( \url{https://autolab.andrew.cmu.edu} ). 

Then login to autolab website, choose \texttt{“15-441: Computer Networks (s19)” -> “project1cp\textit{<N>}”} and then upload your tarball. The grader should be finished in less than a minute but may take longer depending on system load. When it is done, your score will be shown. Only the latest score will be used.\\


Untarring the tarball should give us a directory named \texttt{15-441-project-1} which contains a valid git repo with tags. Your repo should contain the minimum following files:
\begin{itemize}
	\item \textbf{Makefile} -- Make sure all the variables and paths are set
      correctly such that your program compiles in the handin directory---not
      just a local machine or account. The Makefile should, by default, always
      build an executable named lisod.
	\item \textbf{All of your source code} -- (files ending with .c, .h,
      etc. only, no .o files and no executables)
    \item \textbf{readme.txt} -- File containing a brief description of your source tree organization -- what does each file do? This file should also contain your name(s).
	
\end{itemize}

	Late submissions will be handled according to the policy given in the
        course syllabus.


\subsection{Reusing Code}
The code you submit must be your own and we will use tools to compare your code with previous (both from this year and previous years), 
code available on the Internet, etc.  

There are a few exceptions.  You can use the following code:
\begin{itemize}
\item any starter code that we provide (of course!).

\item the following libraries:  \textbf{fill in}

\item any code that was provided to you in the 15-441/641 course.  

\end{itemize}

