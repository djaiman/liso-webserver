\section{Implementation and Submission}
This is a solo project: you \emph{must} implement and submit your own code.  
Source materials for this project may be found on Autolab. 

\subsection{Coding and Version Control}
Your server must be written in the C programming language. You are not allowed
to use any custom socket classes or libraries, only the standard socket library
and the provided library functions. You \textbf{may not} use the csapp wrapper
library from 15-213/15-513 or libpthread for threading. We disallow csapp.c for two
reasons:
first, to ensure that you understand the raw standard BSD sockets API,
and, second, because csapp.c's wrapper functions are not suitable for robust
servers. Temporary system call failures (e.g., EINTR) in functions such as
\texttt{select()} could cause the server to abort and utility functions like
\texttt{rio\_readlineb} are not designed for nonblocking code.

That said, we encourage the use of \emph{anything} for testing.  Use
Wireshark~\cite{wireshark}, use telnet, use real web browsers, use Python to
script tests---for testing, the sky is the limit.

All of your project files and submissions \textbf{must} be stored in a git repository.
%% repository: \url{http://sourcery.cmcl.cs.cmu.edu/indefero}

%%\item All of your project files and submissions \textbf{must} be stored on
%%  AFS in the 441 course directory (\texttt{/afs/andrew/course/15/441-641/}) as a
%%  git-backed repository. You should have a subdirectory in that folder named the
%%  same as your Andrew ID where you must store your git repo (i.e., if your Andrew
%%  ID is \emph{mmukerje} then store your git repo at
%%  \texttt{/afs/andrew/course/15/441-641/mmukerje/}).

You will submit your code as a tarball named \texttt{\textit{<andrewID>}.tar}. Untarring this file should give us a directory named \texttt{15-441-project-1} which should contain the git repository as well as the code. You will submit this tarball using Autolab ( \url{https://autolab.andrew.cmu.edu} ). If you still can't login with your andrew ID by the end of January 22nd, let us know ASAP.\\\\
Checkpoints are designed to ensure that you keep tabs on your progress and are a great guideline to help you complete your project on time. Please note that  Checkpoint 1 is fairly straightforward and a small part of the total work. It helps you to familiarize yourself with the basics of socket programming (specifically how to use \texttt{select()} system call) and  will require some amount of RFC interpretation for implementing a basic HTTP 1.1 parser. The rest of the checkpoints are progressively harder and will build on the previous checkpoints.


\subsection{Compiling}
You are responsible for making sure your code compiles and runs correctly on
the Andrew x86 machines running Linux (i.e., \texttt{linux.andrew.cmu.edu} /
\texttt{unix.andrew.cmu.edu}). We recommend using \texttt{gcc} to compile your
program and \texttt{gdb} to debug it. You should use the \texttt{-Wall} and
\texttt{-Werror} flags when compiling to generate full warnings and to help
debug. Other tools available on the Andrew unix machines that are suggested are
ElectricFence~\cite{WWW:efence} (link with \texttt{-lefence}) and
Valgrind~\cite{WWW:Valgrind}---use this with full leak checking to ensure you
have no memory leaks.  For this project, \textbf{you will also be responsible
for turning in a GNU Make compatible Makefile}. See the GNU make
manual\cite{Manual:Make} for details. When we run \texttt{make} we should end
up with the Liso web server binary \textbf{\texttt{lisod}}.
\subsection{Running}
%This is how we will start your implementation of Liso:\\
%\emph{
%\indent./lisod 8080 4443 /tmp/lisod.log /tmp/liso.lock /tmp/www /tmp/cgi/cgi_script.py /tmp/priv.key /tmp/cert.crt \\
%}
The Liso server will be passed the ports to run on, what log file to use, what
lock file to use when daemonizing, folders to serve static data from as well
as CGI applications, and TLS private/public key pairs.

Not all of these options need to be functional at each stage of development.
Only a HTTP port is needed for the first checkpoint when implementing an echo server
using \texttt{select()}.

\subsection{Framework Code} 
We will provide you with framework code that will, for example, help in forking
a process for proper CGI handling and setting up the environment, parse
commandline arguments (and sanity check them) and daemonize a process.

\textbf{DISCLAIMER:} We reserve the right to change the support code as the
project progresses to fix bugs and to introduce new features that will help you
debug your code. You are responsible for checking Piazza to stay
up-to-date on these changes. We will assume that all students in the class will
read and be aware of any information posted to Piazza.

\subsection{HTTP Packet Parsing}
Historical evidence suggests that most students spend considerable
amount of time writing correct parsers. While parsing packets using
C's string manipulation functions may well be an essential skill to
have, it might get insanely tedious. We want you to spend time on
other more important programming aspects such as socket programming,
handling race conditions and memory leaks. For this reason, we require you to use Lex and Yacc for parsing packets. We will also provide you with a basic HTTP parser written in lex and yacc. More about parsing
using Lex and Yacc will be covered in recitations. So, stay tuned!


\subsection{Hand-In}
\label{sec:handin}
Handing in code for checkpoints and the final submission deadline will be done
through Autolab (\url{https://autolab.andrew.cmu.edu}).  You are supposed to upload the tarball file for each checkpoint into the corresponding assessment on our course page of Autolab website.

\subsection{Work with git}
You are supposed to create your git repo on your local machine or on a \textbf{private} git repo hosted
online as part of
Checkpoint 1. Every checkpoint will be a git tag in this repo. To create a tag, run
\[
\texttt{git tag -a checkpoint-\textit{<num>} -m \textit{<message>} [\textit{<commit hash>}]}
\]
with appropriate checkpoint number and custom message filled in. (Put whatever
you like for the message --- git won't let you omit it.) The optional commit
hash can be used to specify a particular commit for the tag; if you omit it,
the current commit is used. If you choose to clone your repository onto your
local machine for development, be sure to use \texttt{git push --tags} to sync
your work back to git server; the standard \texttt{git push} doesn't send tags.

\subsection{Upload your code}
To submit your code, make a tarball file of you repo after you tag it. Then login to autolab website, choose \texttt{“15-441: Computer Networks (s19)” -> “project1cp\textit{<N>}”} and then upload your tarball. The grader should be finished in less than a minute but may take longer depending on system load. When it is done, your score will be shown. Only the latest score will be used.\\


Untarring the tarball should give us a directory named \texttt{15-441-project-1} which contains a valid git repo with tags. Your repo should contain the minimum following files:
\begin{itemize}
	\item \textbf{Makefile} -- Make sure all the variables and paths are set
      correctly such that your program compiles in the handin directory---not
      just a local machine or account. The Makefile should, by default, always
      build an executable named lisod.
	\item \textbf{All of your source code} -- (files ending with .c, .h,
      etc. only, no .o files and no executables)
	\item \textbf{readme.txt} -- File containing a brief description of your source tree organization.
	
	\item \textbf{tests.txt} -- File containing documentation of your test cases
      and any known issues you have.
%    \item \textbf{replay.test} -- File containing input to send to the server
%      testing the implementation
%    \item \textbf{replay.out} -- File containing expected server output matching
%      input given in replay.test
    %\item %\textbf{vulnerabilities.txt} -- File containing documentation of at
    %  least one vulnerability you identify at each stage.
\end{itemize}

	Late submissions will be handled according to the policy given in the
        course syllabus.

%\newpage


