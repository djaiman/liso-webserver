In this class you wil learn about Computer Networks, from bits on a wire or radio (`the bottom') all the way up to the design of applications that run over networks (`the top)'. 
In this project, you will start from the top with an application we are all familiar with: a web server.

You will use the Berkeley Sockets API to write a web
server using a subset of the HyperText Transport Protocol (HTTP) $1.1$ ---RFC 2616~\cite{httprfc}.  Your web server will also implement
HyperText Transport Protocol Secure (HTTPS) via Transport Layer Security (TLS)
as described in RFC 2818~\cite{httpsrfc}.  The final part of the project will
implement the Common Gateway Interface (CGI) as described in RFC
3875~\cite{cgirfc}.  

RFCs are one of many ways the Internet declares `standards:' agreed upon algorithms, wire formats, and protocols for interoperability between different implementations of systems on a network.
Without standards, software from one company would not be able to talk to software from another company and we would not have things like e-mail, the Web, or even the ability to route traffic between different companies or countries.
Because your web server is compatible with these standards, you will, by the end of this project, be able to browse the content on your web server using any standard browser like Chrome or Firefox.
 
\vspace{5pt}

\noindent With this project, you will gain experience in:
\begin{itemize}
  \item ... building non-trivial computer systems in a low-level language (C).
  \item ... complying with real industry standards as all developers of Internet services do.
  \item ... reasoning about the many tasks that application developers rely on the Internet to do for them.
\end{itemize}

\noindent To guide your development process, we have divided this project into three key checkpoints.
